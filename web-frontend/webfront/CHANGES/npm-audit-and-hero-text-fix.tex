% npm-audit-and-hero-text-fix.tex
\documentclass{article}
\usepackage[utf8]{inputenc}
\usepackage{hyperref}
\title{Security Audit and UI Readability Fix}
\author{Automated Assistant}
\date{2025-09-20}

\begin{document}
\maketitle

\section*{Summary}
This document summarizes two recent maintenance actions applied to the `propestateai-webfront` project (located at `web-frontend/webfront`):
\begin{enumerate}
  \item A dependency audit using `npm audit` identified several transitive vulnerabilities originating from `react-scripts` and its dependencies (e.g., `svgo`, `@svgr/*`, `nth-check`, `css-select`, `postcss`, `webpack-dev-server`).
  \item A UI readability fix for the homepage "demo" sentence was added to improve contrast on the hero background image.
\end{enumerate}

\section*{npm audit findings}
The `npm audit --json` report found 9 vulnerabilities (3 moderate, 6 high). Key affected packages included:
\begin{itemize}
  \item `react-scripts@5.0.1` (direct dependency) --- pulls in older transitive packages with advisories.
  \item `svgo`, `@svgr/plugin-svgo`, `@svgr/webpack` --- high severity issues.
  \item `nth-check`, `css-select` --- high severity transitive issues via `svgo`.
  \item `postcss`, `resolve-url-loader`, `webpack-dev-server` --- moderate severity issues.
\end{itemize}

\section*{Recommended remediation steps}
\begin{enumerate}
  \item Run `npm audit fix` to apply non-breaking upgrades automatically.
  \item If necessary and after testing, consider `npm audit fix --force` (may introduce breaking upgrades).
  \item For a robust long-term solution, migrate away from `react-scripts` (Create React App) to a modern toolchain such as Vite which has fewer legacy transitive dependencies.
  \item Always create a Git branch and commit current `package.json` and `package-lock.json` before applying global upgrades so changes can be reviewed and reverted if necessary.
\end{enumerate}

\section*{UI fix applied}
The homepage was updated to include the following paragraph text and styling to improve readability over the hero background image:
\begin{verbatim}
A demo web UI — chat with the agent, get spoken responses, and explore listings.
\end{verbatim}

The change was implemented in `web-frontend/webfront/src/App.js` as an inline-styled paragraph with white color and a dark text-shadow. The mobile counterpart was updated in `mobile-frontend/src/styles.css` to force the `.muted` paragraph to white and add a dark shadow:
\begin{verbatim}
.hero-card .muted { color: #fff !important; text-shadow: 0 2px 8px rgba(0,0,0,0.85) !important; }
\end{verbatim}

\section*{Verification steps}
\begin{enumerate}
  \item In the web frontend:
    \begin{verbatim}
    cd web-frontend/webfront
    npm install
    npm start
    \end{verbatim}
    Open the browser at the Vite/CRA dev server address (commonly http://localhost:3000 or the printed dev server URL) and confirm the new demo sentence appears in white with a dark shadow.
  \item In the mobile frontend:
    \begin{verbatim}
    cd mobile-frontend
    npm install
    npm run dev
    \end{verbatim}
    Open the mobile dev server address (commonly http://localhost:5173) and confirm the `.muted` paragraph on the Home hero appears white with a dark shadow.
\end{enumerate}

\section*{Change log}
\begin{itemize}
  \item `web-frontend/webfront/src/App.js` — inserted demo paragraph with inline white color and text-shadow.
  \item `mobile-frontend/src/styles.css` — updated `.hero-card .muted` rule to white with dark text-shadow.
  \item `mobile-frontend/CHANGES/muted-paragraph-readability.txt` — added explanatory changelog note.
\end{itemize}

\end{document}
